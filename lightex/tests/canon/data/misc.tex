\newcommand{\xxx}{\xxx!}

asdf
\begin{itemize}
    \item 1\\
    \item 2
    \item 3
\end{itemize}
asdf
\begin{enumerate}
    \item 1\\
    \item 2
    \item 3
\end{enumerate}
asdf

% \xxx

$\set{a, b}$

$\setminus$

Нанокристалл стабилен, если между любыми двумя его атомами можно построить 
соединяющую их цепочку связей, возможно с использованием других атомов. 
Например, нанокристалл $X$ из четырёх атомов $A$, $B$, $C$ и $D$, в котором между собой 
связаны пары $\{\,A, B\,\}$, $\{\,A, C\,\}$, $\{\,B, C\,\}$ и $\{\,B, D\,\}$, стабилен.
Если же, например, в нанокристалле из данных четырех атомов связаны только пары

426~$\to$~4 6~$\to$~456~$:$~456\\

$f\colon X \to Y$.

$a \le b$.

\textit{фыва} фыва!

Входной файл содержит описание города и плана эвакуации.
Первая строка содержит два числа $N$ и $M$, разделённых пробелом~--- число 
муниципальных зданий в городе (все они пронумерованы от 1 до $N$)
и число убежищ (все они пронумерованы от 1 до $M$) соответственно ($1 \le N, M \le 100$).

Имеется $N$ прямоугольных конвертов и $N$ прямоугольных открыток различных 
размеров. Необходимо определить, можно ли разложить все открытки по конвертам,
чтобы в каждом конверте было по одной открытке. Открытки нельзя 
складывать, сгибать \mbox{и~т.\,п.,} но можно помещать в конверт под углом. Например, 
открытка с размерами сторон $5:1$ помещается в конверты с размерами $5:1$, $6{:}3$, 
$4{,}3:4{,}3$, но не входит в конверты с размерами $4:1$, $10:0{,}5$, $4{,}2:4{,}2$.

$1 \oslash 1$

$10^{ - 4 + 3}$

На плоскости расположено $N$ точек. Имеется робот, который двигается следующим 
образом. Стартуя с некоторой начальной точки и имея некоторое начальное 
направление, робот движется до первой встреченной на его пути точки, изменяя в 
ней свое текущее направление на $90^\circ$, \mbox{т.\,е.} поворачивая налево или 
направо. После этого он продолжает движение аналогичным образом. Если робот 
достиг начальной точки, либо не может достичь новой точки (которую он ещё не 
посещал), то он останавливается.
Необходимо определить, может ли робот посетить все $N$ точек, если определены 
начальные точка и направление движения робота (при этом робот должен 
обязательно вернуться в начальную точку).

a$:$a

$1 \oplus 1 \otimes 1 \ominus 1 \odot 1$

$- -- --- ---- < > << >> <<< >>>$

$\circ$

$90^\circ + 1$

$90^\ldots + 90^{ax} + 90$

$3^{3^{3^{3^{3^{3^{3^{3^{3^{3^{3^{3^{3^{3^{3^{3^{3^{3^{3^{3^{3^{3^3}}}}}}}}}}}}}}}}}}}}}$

$90^{\circ}$
