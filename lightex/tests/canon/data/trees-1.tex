\begin{rawproblem}{in.txt}{out.txt}

Найти среднюю по значению вершину из вершин дерева, у которых число потомков в левом поддереве
не равно числу потомков в правом поддерева.
Удалить её (правым удалением), если такая вершина существует.
Выполнить прямой левый обход полученного дерева.

\InputFile

Входной файл содержит последовательность чисел~--- ключи вершин в порядке добавления в дерево.

\OutputFile

Выходной файл должен содержать последовательность ключей вершин, полученную прямым левым обходом итогового дерева.

\Example

\begin{example}%
\exmp{%
20
8
40
42
14
4
13
41
5
1
}{%
40
8
4
1
5
14
13
42
41
}%
\end{example}%
\begin{center}%
    \includegraphics{static/fig1.png}\includegraphics{static/fig2.png}%
\end{center}%

\Note

В случае неоднозначности выбора удаляемой вершины (например, несколько путей 
максимальной длины между вершинами с разным числом потомков и с минимальной 
суммой ключей конечных вершин имеют один и тот же корень, но средние по 
значению вершины этих путей не совпадают) ничего из дерева удалять не нужно.

\end{rawproblem}
