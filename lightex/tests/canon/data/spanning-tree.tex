\begin{rawproblem}{стандартный ввод}{стандартный вывод}
Задан неориентированный граф без петель и кратных ребер.

Требуется построить какое-либо остовное дерево этого графа или сообщить, что его не существует.

\InputFile
В первой строке входного файла записано целое число $N$~--- количество вершин в графе ($1 \le N \le 100$). Далее записана матрица смежности графа: $N$ строк по $N$ чисел, каждое из которых~--- $0$ или $1$. Числа разделяются одиночными пробелами. Гарантируется, что матрица симметрическая, все элементы на главной диагонали нулевые.

\OutputFile
Если остовное дерево построить невозможно, выведите одно число $-1$. Иначе в первой строке выведите число $M$~--- количество рёбер в остовном дереве, в последующих $M$ строках напечатайте сами рёбра, т.~е. укажите для каждого ребра пару вершин, которые это ребро соединяет. Вершины графа нумеруются числами $1$, $2$, \ldots, $N$. Порядок вывода рёбер значения не имеет. Если остовных деревьев несколько, выведите любое.

\Examples
\begin{example}
\exmp{3
0 1 0
1 0 1
0 1 0
}{2
1 2
3 2
}%
\exmp{2
0 0
0 0
}{-1
}%
\end{example}

\end{rawproblem}
