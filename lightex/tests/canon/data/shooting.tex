\begin{rawproblem}{shooting.in}{shooting.out}

На соревнованиях по стрельбе каждый участник будет стрелять в цель, которая 
представляет собой прямоугольник, разделенный на квадраты. Цель содержит $R\cdot C$
квадратов, расположенных в $R$ строках и $C$ столбцах.
Квадраты выкрашены в белый или чёрный цвет.
В~каждом столбце находится ровно 2 белых и $R-2$ чёрных квадрата.
Строки пронумерованы от 1 до $R$ сверху вниз, а столбцы~--- от 1 до $C$ слева 
направо.
Стрелок имеет ровно $C$ стрел.
Последовательность из $C$ выстрелов называется \textit{корректной},
если в каждом столбце поражён ровно один белый квадрат,
а в каждой строке~--- не менее одного белого квадрата.

Необходимо проверить, 
существует ли корректная последовательность выстрелов, и если да, то найти одну 
из них.

\InputFile

Первая строка содержит два целых числа $R$ и $C$,
разделённых пробелом ($2 \le R \le C \le 1000$).
Эти числа определяют количество строк и столбцов.

Каждая из следующих $C$ строк в блоке содержит два натуральных числа,
разделённых пробелом.
Числа в $(i+1)$-й строке определяют номера строк, где 
расположены белые квадраты в $i$-м столбце.

\OutputFile

Выведите последовательность из $C$ номеров строк (разделённых пробелом)
корректной последовательности выстрелов в белые 
клетки столбцов $1, 2, \ldots, C$ соответственно или сообщение \texttt{No}, если такой 
последовательности не существует.

\Example

\begin{example}%
\exmp{% input
4 4
2 4
3 4
1 3
1 4
}{% output
2 3 1 4
}%
\end{example}

\end{rawproblem}
