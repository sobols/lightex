\begin{document}
\begin{rawproblem}{input.txt}{output.txt}

Доминошка~--- это прямоугольная плитка, лицевая сторона которой разделена на
два квадрата, каждый из которых содержит от 0 до 6 точек.
Ряд доминошек выложен на столе:

\begin{center}
\begin{verbatim}
6 1 1 1
- - - -
1 5 3 2
\end{verbatim}
\end{center}

Число точек в верхней строке равно $6 + 1 + 1 + 1 = 9$, а в
нижней~--- $1 + 5 + 3 + 2 = 11$.
Разница между нижней и верхней строкой составляет $|11 - 9| = 2$.
Разница~--- это абсолютная величина разности двух сумм.
Каждая доминошка может быть повёрнута на $180^{\circ}$,
меняя местами верхний и нижний квадраты.

Необходимо определить, какое минимальное число поворотов необходимо для
минимизации разницы.
В~приведённом примере нужно повернуть последнюю доминошку для того, чтобы
уменьшить разницу до нуля.
В~этом случае ответом будет~1.

\InputFile

Первая строка содержит число $n$ доминошек, лежащих на столе ($1 \le n \le 250\,000$).
Каждая из следующих $n$~строк содержит по два целых числа $a$ и $b$, разделённых
пробелом ($0 \le a, b \le 6$).
Числа $a$ и $b$, записанные в $(i + 1)$-й строке определяют число точек на
$i$-й доминошке в верхнем и нижнем квадратах соответственно.

\OutputFile

Выведите наименьшее число поворотов, необходимых для минимизации разности.

\Example

\begin{example}%
\exmp{% input
4
6 1
1 5
1 3
1 2
}{% output
1
}%
\end{example}


\end{rawproblem}
\end{document}
